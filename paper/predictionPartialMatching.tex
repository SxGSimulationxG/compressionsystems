\subsection{Prediction with Partial Matching}
Die Methode der "Prediction with Partial Matching" verwendet einen Suchbaum, der als "Trie" bezeichnet wird, mit einer maximalen Tiefe $D$, wobei jeder Knoten eine bestimmte Wahrscheinlichkeitsverteilung in Abhängigkeit von seinen Eltern sowie Vine-Zeiger auf die darüber liegende Ebene aufweist. Diese Methode funktioniert in mehreren Schritten\footnote{\cite{steinruecken2015lossless} p.111-113: 6.3.1 Basic Operation}:
\begin{enumerate}
	\item Erstellen Sie einen leeren Baum.
	\item Wiederholen Sie dies, bis das Ende erreicht ist:
	\subitem Lesen Sie die nächste Eingabe.
	\subitem Verwenden Sie die Wahrscheinlichkeitsverteilung des aktuellen Knotens, um die Eingabe zu codieren.
	\subitem Aktualisieren Sie die Wahrscheinlichkeitsverteilung und den entsprechenden Vine-Knoten.
	\subitem Suchen Sie den erforderlichen Knoten, der der Eingabe entspricht. Wenn die Tiefe maximal ist, wechseln Sie zum Knoten des Vine-Zeigers. Wenn der erforderliche Knoten nicht existiert, erstellen Sie einen und aktualisieren Sie den Baum, damit er funktioniert.
	\subitem Setzen Sie einen Zeiger vom aktuellen Knoten zum Knoten des Kindes und gehen Sie zu diesem Knoten.
\end{enumerate}
Dies lässt sich grafisch darstellen, wobei die gepunkteten Linien die Vine-Zeiger und die normalen Zeiger den Pfad anzeigen, den der Algorithmus durchlaufen hat. Der schattierte Knoten ist die Position am Ende des Algorithmus:

\begin{figure}[ht]
	\centering
	\includegraphics[width=1.1\linewidth]{ppm}
	\caption{Beispiele aus \cite{steinruecken2015lossless}}
	\label{fig:ppm}
\end{figure}