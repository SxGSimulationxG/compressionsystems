\section{Schlusswort}
Auch wenn die ursprünglichen Komprimierungssysteme von Huffman, Shannon und Fano heute nicht mehr allzu frequent in ihrer ursprünglichen Form verwendet werden, bilden sie doch die Grundlage für moderne verlustfreie Komprimierungssysteme. Huffman zeigt uns, dass es in einem zeroth-order Komprimierungssystem einen optimalen Komprimierungsalgorithmus gibt, der mit einem Bottom-up-Konstruktionssystem funktioniert. Die Kodierungssysteme von Fano und Shannon sind zwar nicht so effektiv, wenn es darum geht, sich dem Wert der Entropie anzunähern, aber sie bieten einen intuitiveren Überblick und eine intuitivere Methode zur Implementierung eines verlustfreien Kompressionsalgorithmus. Anhand der verwendeten Beispiele können wir jedoch sehen, wie stark sich diese Verfahren unterscheiden. Wenn wir uns higher-order Modelle ansehen, werden die Algorithmen viel komplizierter und speichereffizienter, aber am Ende kommt es fast immer zu einem zeroth-order Komprimierungssystem, nur dass die Eingabe effektiver komprimiert werden kann als die ursprüngliche Eingabe.

