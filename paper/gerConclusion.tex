\section{Schluss}
Auch wenn die ursprünglichen Komprimierungssysteme von Huffman, Shannon und Fano heute nicht mehr so häufig in ihrer ursprünglichen Form verwendet werden, bilden sie doch die Grundlage für moderne verlustfreie Komprimierungssysteme. Huffman zeigt uns, dass es in einem zeroth-order Komprimierungssystem einen optimalen Komprimierungsalgorithmus gibt, der mit einem Bottom-up-Konstruktionssystem funktioniert. Die Kodierungssysteme von Fano und Shannon sind zwar nicht so effektiv, wenn es darum geht, sich dem Wert der Entropie anzunähern, aber sie bieten einen intuitiveren Überblick und eine intuitivere Methode zur Implementierung eines verlustfreien Kompressionsalgorithmus. Anhand der verwendeten Beispiele können wir jedoch sehen, dass der Unterschied zwischen ihren Kompressionsgrößen zunächst recht gering ist, aber mit zunehmender Dateigröße zunimmt. In diesem Beispiel können wir auch sehen, wie Shannons Theorie funktioniert: Je größer die Datei wird, desto mehr Speicherplatz wird eingespart. Wenn wir uns higher-order Modelle ansehen, können die Dinge viel komplizierter und speichereffizienter werden, aber am Ende kommt es fast immer zu einem zeroth-order Komprimierungssystem, nur dass die Eingabe effektiver komprimiert werden kann als die ursprüngliche Eingabe.

