\documentclass[a4paper,12pt]{article}
\usepackage{graphicx} % Required for inserting images
\usepackage{listings}
\lstset{breaklines=true}
\usepackage{todonotes}
\usepackage{biblatex}
\usepackage{amsmath}
\usepackage{setspace}
\usepackage[a4paper,margin=2.5cm]{geometry}
\setlength{\marginparwidth}{2cm}
\addbibresource{main.bib}

% --- Grafiken & Bilder ---
\usepackage{graphicx}
\graphicspath{{images/}}
\usepackage{float}

% --- Tabellen ---
\usepackage{array}
\usepackage{booktabs}
\usepackage{multirow}
\usepackage{setspace} 	% zum Zeilenabstand setzen
\usepackage{marvosym} 	%div. Symbole
\usepackage{multicol} 	% analog s.o.
\usepackage{booktabs} 	% dünne und dicke Striche bei Tabellen
\usepackage{diagbox}
\usepackage{tablefootnote}

\usepackage{tabularx}
\newcolumntype{L}[1]{>{\raggedright\arraybackslash}p{#1}} % linksbündig mit Breitenangabe
\newcolumntype{C}[1]{>{\centering\arraybackslash}p{#1}} % zentriert mit Breitenangabe
\newcolumntype{R}[1]{>{\raggedleft\arraybackslash}p{#1}} % rechtsbündig mit Breitenangabe
\newcommand{\ltab}{\raggedright\arraybackslash} % Tabellenabschnitt linksbündig
\newcommand{\ctab}{\centering\arraybackslash} % Tabellenabschnitt zentriert
\newcommand{\rtab}{\raggedleft\arraybackslash} % Tabellenabschnitt rechtsbündig

\usepackage{hhline}
\usepackage{colortbl}

% \usepackage[table]{xcolor} % for \cellcolor

% Optional color definitions
\definecolor{hellgrau}{RGB}{230,230,230}
\definecolor{grau}{RGB}{200,200,200}

\addtolength{\footskip}{-4mm} % Seitenzahl vertikal verschieben.

\usepackage{fancyhdr}
\pagestyle{fancy}

% Clear defaults
\fancyhf{}

% Header offset beyond the right margin (e.g., 1cm)
\fancyheadoffset[R]{1.5cm}

% Define header content (example)
\fancyhead[L]{\nouppercase{\leftmark}}  % right-aligned section title
\fancyhead[R]{\thepage}


\begin{document}
	
	\newgeometry{left=2cm, right=2cm,}
	\begin{titlepage}
	
	\newcommand{\HRule}{\rule{\linewidth}{0.4mm}} % Line definition
	\centering
	
	%----------------------------------------------------------------------  
	%  Header
	%----------------------------------------------------------------------  
	\textsc{\LARGE Friedrich-Koenig-Gymnasium Würzburg}\\[1cm]
	\includegraphics[scale=0.6]{fkg-logo_06-2024.0x80.jpg}\\[0.8cm]
	\textsc{\Large W-Seminar 2024 | 2026}\\[0.5cm]
	\large Mathematik\\[0.5cm]
	
	%----------------------------------------------------------------------  
	%  Title
	%----------------------------------------------------------------------  
	\HRule \\[0.5cm]
	{ \Large \bfseries On The Optimality of Huffman and Shannon-Fano Coding and Their Subtypes\\[0.2cm]
		\large \mdseries {}}
	\HRule \\[1cm]
	
	%----------------------------------------------------------------------  
	%  Authors
	%----------------------------------------------------------------------  
	\begin{minipage}{0.45\textwidth}
		\begin{flushleft} \large
			\emph{Lehrkraft:}\\
			Oliver \textsc{Manger}
		\end{flushleft}
	\end{minipage}
	\hfill
	\begin{minipage}{0.45\textwidth}
		\begin{flushright} \large
			\emph{Schüler:}\\
			Kai \textsc{Rasmussen}
		\end{flushright}
	\end{minipage}\\[1.5cm]
	
	%----------------------------------------------------------------------  
	%  Table
	%----------------------------------------------------------------------  
	\setlength{\extrarowheight}{0pt}
	\setlength{\arrayrulewidth}{0.4pt}
	
	\normalsize
	\begin{tabularx}{\textwidth}{l|c|p{6.3cm}|c|c|c}
		\toprule
		\textbf{Bewertung}  & \textbf{Note} & \textbf{Notenstufe in Worten} & \textbf{Punkte} & & \textbf{Punkte} \\
		\midrule
		Schriftliche Arbeit & & & & $\times$ 3 & {\cellcolor{hellgrau}\strut} \\ 
		\midrule
		Präsentation & & & & $\times$ 1 & {\cellcolor{hellgrau}\strut} \\ 
		\midrule
		\multicolumn{5}{r|}{\textbf{Summe:}} & {\cellcolor{hellgrau}\strut} \\ 
		\midrule\midrule
		\multicolumn{5}{r|}{\textbf{Gesamtleistung} {\small nach § 61 (7) GSO = Summe : 2 (gerundet)}:} & {\cellcolor{grau}\strut} \\
		\bottomrule
	\end{tabularx}
	
	\vspace{\fill}
	
	\begin{tabular}{cp{2em}c} 
		\hspace{7cm}   && \hspace{7cm} \\\cline{1-1}\cline{3-3} 
		\textit{Abgabe im Sekretariat} && \textit{Unterschrift des Kursleiters} 
	\end{tabular}
	
\end{titlepage}
	\restoregeometry
	
	\tableofcontents
	\newpage
	
	\textbf{Abstract - The comparison between the original data compression algorithms, Huffman and Shannon-Fano Coding systems, as well as a deeper look into the different improvements to and applications of lossless compression algorithms.}\\
	
	Keywords: Huffman coding, Shannon coding, Fano coding, lossless data compression, static compression systems, adaptive compression systems, and higher-order models.
	
	\section{Introduction}

Lossless data compression systems have a massive importance to our modern lives. Without these systems we couldn't have managed to reach this level of the technological era. As the name says, these are systems to reduce the amount of storage data takes up without loosing any information in the process, i.e. lossless. The foundation of lossless data compression systems are Huffman, Shannon, and Fano coding systems but since these systems were introduced there have been many improvements and different approaches to this problem of lossless data compression. 

\subsection{Aims}
In this paper we will firstly discuss the history of data compression leading up to the invention of the fundamental lossless compression systems and we will then go through these systems, going through how they work and how they differ from each other. We will then go into the difference between their static and adaptive counterparts before briefly looking at how these systems were improved upon and how they are used in the modern day. Afterwards we will compare the fundamental lossless compression systems using example data with a java comparison system to find out by how much these systems differ and if there are certain biases with the data examples.

\subsection{Theoretical Basis}
The purpose of this section is as a reference during the reading of the rest of the paper for any base concepts.

\subsubsection{Lossless vs. Lossy Compression Systems}
Lossless compression systems are a method of compression in which the original uncompressed form can be reconstructed to it's original quality. In comparison lossy compression is a system where less relevant information is disregarded to compress the size\footnote{\cite{blelloch2001introduction} p.40-41: 7 Lossy Compression Techniques}. In this paper we will only be looking at lossless compression systems.

\subsubsection{Variable vs. Fixed Length Coding}
Fixed length coding systems assign a certain amount of bits per character, each character being the same length. This makes it easier to search for a specific position of a character through simple multiplication, i.e. if you were looking for the 10th character in an 8-bit fixed length coding system you would look from bit 81 to 88. Variable length coding systems however, as the name tells us, the character length vary. This is then helpful when we want to compress data so we can assign shorter codes to more common characters\footnote{\cite{vitter1987design} p.825-827: 1 Introduction}.

\subsubsection{Information Entropy}
Information entropy describes the relation between the probabilities of the input sequence and the length of the code word. For the most optimal lossless compression system, the average code word length is equal to the value of the information entropy. This concept was one of the most important in Shannon's foundation of information theory\footnote{\cite{shannonMathematicalCommunication} p.396-399: 7 The Entropy of an Information Source}. The moment when the average code word length goes below the value of the entropy $H$, information must be lost, so the compression system is no longer lossless.

\begin{equation}
	H=-\sum p_i \log(p_i)
\end{equation}

This equation is valid when the characters are independent of each other.

\subsubsection{Higher-order vs. Zeroth-order Systems}
In a zeroth-order system, each character is encoded independently, any correlation between the characters in the input sequence aren't taken into account. A higher-order system has the ability to observe and utilize patterns across multiple characters.

\subsubsection{Static vs. Adaptive Coding Systems}
In a static coding system, we have knowledge about the probabilities of the source sequence before encoding the sequence. In an adaptive system the only input the encoder has is the source sequence itself. The system then adapts the encoding scheme with changes in the frequency of the input characters\footnote{\cite{rueda2002advances} p.60-61: 2.5 Adaptive Coding}.

\subsubsection{Dictionary Based Coding}
Storer and Szymanski in their paper "Data Compression via Textual Substitution" lay a foundation of dictionary based data compression. Their system adds several components: external macro schemes, internal macro schemes, a dictionary, and a skeleton. The external macro scheme allows a source string to be encoded using a dictionary, a storage of reference strings, and a skeleton, a combination of characters of the input alphabet and pointers to the dictionary. The internal macro schemes allow pointers to duplicate sections of the same string\footnote{\cite{storer1982data} p.929-932: 2 The Model and Basic Definitions}. With this system we can decrees the amount of redundancy in our input sequence. 

\subsubsection{Dynamic Markov Compression}

\subsubsection{Block Sorting Compression}
The block sorting compression algorithm, also known as the Burrows-Wheeler Transform (BWT) is a method of sorting an input sequence into a form more desirable to compress using other methods. We start with an input sequence $\mathcal{X} = \{x_1 \to x_M\}$ and create a $M \times M$ matrix. Each row of the matrix being the input sequence shifted by the one from the row before. We then take the matrix and sort it alphabetically, this matrix we then call $A$. The output of the transform being the last column of $A$, that we call $T$, and the row where $\mathcal{X}$ can be found, which we will call $k$. When reverting the transformation we sort $T$ alphabetically in a new list $L$, noting the position of the character in $L$ in $T$ in a new list $F$. Using $F$ we can reconstruct the original sequence $\mathcal{X}$\footnote{\cite{rueda2002advances} p.79-80: 2.6.5 Block Sorting Compression; \cite{steinruecken2015lossless} p.33-34: Block Compression}.

\subsubsection{Grammar Based Coding}
In grammar based coding, we use the concept of languages and grammars in computer science, a grammar $G$ being made out of a list of terminals $T$ or an alphabet $\Sigma$, a list of non-terminals $N$, a list of production rules $P$, and a start node $s$. With this higher-order system instead of encoding the string $x$ itself, we encode the grammar $G_x$. Due to the fact that the grammar only constructs the string x, it is called a context-free grammar\footnote{\cite{kieffer2000grammar} p.2-4: Introduction}.

\begin{equation}
	L(G_x)=\{x\}
\end{equation}

In this context, the aim is to make the smallest possible context-free grammar for a certain string $x$. The size of a grammar $|G_x|$ being the amount of symbols on the right side of the production rules $P$. This problem in question is called the "Smallest Grammar Problem" (SGP)\footnote{\cite{charikar2005smallest} p.1-2: Introduction}. The example given in \cite{charikar2005smallest} being:

\begin{equation}
	P=\{<S> \to <B><B><A>,\\
	<A> \to \text{a rose},\\
	<B> \to <A> \text{ is }\}
\end{equation}

Here $|G_x|=14$ being the smallest grammar for making the string "a rose is a rose is a rose".

\subsubsection{Prediction with Partial Matching}
The method of prediction with partial matching uses a search tree called a "trie" with a maximum depth $D$, each node having a certain probability distribution depending on it's parents as well as vine pointers to the layer above. This method works in several steps:
\begin{enumerate}
	\item Create an empty tree.
	\item Repeat until end is reached:
	\subitem Read the next input.
	\subitem Use the probability distribution of the current node to encode the input.
	\subitem Update the probability distribution and the corresponding vine node.
	\subitem Find the required node corresponding to the input from the row above, if this doesn't exist create one and update the tree to make it work.
	\subitem Set a pointer from the current node to the node of the child and move to that node.
\end{enumerate}
This can be visualised graphically, here the dotted lines indicating the vine pointers and the normal pointers the path in which the algorithm passed through, the shaded node is the position at the end of the algorithm:



	\section{History of Data Compression}
In general, a data compression system is used when we want to compact a larger amount of data into a smaller amount of information and as a result also decrease the amount of time required to transfer messages. This principle can already be seen in the 4th century BCE in Greece where Polybius tells us of an improved communication system introduced by Aeneas Tacticus, in which two people in viewing distance would use a torch and two identical inscribed vessels to communicate at faster speeds in times of war\footnote{\cite{polybiusHistories} Book X pp.42-43: 44. The Improvement introduced by Aeneas Tacticus}. Here they used water and engravings in the vessel, a certain volume of water being a specific message. This is an early example of a cryptographic cypher, where a certain input, here the volume of water, can be decrypted to a certain plain text message\footnote{\cite{cnssGlossary} p.32: Cipher (American English)}. In the same vane, many versions of the optical semaphore were used in a similar manner, using predefined interpretations of certain positions to communicate a predetermined message. However, one of the flaws of both of these systems is that there is a limit to what information you can transmit as you cannot fit every possible message into a cypher. Communicating a single letter at a time would take a significant amount of time so to reduce the quantity of data being sent but retain all the information required to produce any message, the idea of lossless data compression was born. The first main compression algorithm to come up was Claude Shannon's coding system, named after himself (See Section 3.2: Shannon Coding) based on his noiseless coding theorem. This theorem and compression algorithm make the foundations of what would become the field of information theory\footnote{\cite{rueda2002advances} p.36: 2.2.1 Introduction to Information Theory}. 
At roughly the same time Robert Fano created what would become known as Fano coding. Even though Shannon's and Fano's coding systems are fundamentally different even if they do function in roughly the same manner, they are often combined into Shannon-Fano coding, mostly referring to Fano's coding system. These systems however are both suboptimal when it comes to the resulting compression size\footnote{\cite{krajvci2015performance} p.1: Introduction}. 
It all then changed when David A. Huffman attended an electrical engineering graduate course on information theory taught by Robert Fano. In this course the students were given the choice between a term paper or a final exam. The topic of the term paper was to find the most optimal lossless compression system for a binary system, the very thing that Fano and Shannon were trying to achieve themselves. Huffman decided to take on the challenge and write the term paper, writing one of the most influential papers in the field, "A method for the construction of minimum-redundancy codes" as a result. This he proved was the most optimal system to construct such a lossless compression system\footnote{\cite{pivkinadiscovery}: p.1: Introduction}.
Since then there have been many improvements to the field, some of which we will discuss in the section on higher-order systems and in connection with the adaptive versions of these original systems, however these improvements vary widely and it would be impossible to showcase all of them. 

\todo{Image of graphic of hydraulic telegraph}
\todo{To history of shannon and fano systems}
\todo{To Huffman}
\todo{To modern extensions and versions (Out of scope)}
	\section{Compression Systems}
In this paper, when explaining and comparing these systems, we will follow a specific notation and terminology in the creation of our compression tree. As the input for any compression system, we need an input sequence $\mathcal{S}$ which as an array of $m$ unique characters. Each of these characters will then, through the compression system, receive their own code. 
\begin{equation}
	\mathcal{S}=\{s_1 \to s_m\}
\end{equation}

Each of these characters in the input sequence has an attached probability. This probability can be calculated by dividing the amount of appearances of this character by the total amount of characters. These probabilities are also kept in an array $\mathcal{P}$.
\begin{equation}
	\mathcal{P}=\{p_1 \to p_m\}
\end{equation}
\begin{equation}
	p=\frac{\textbf{Amount of appearances}}{\textbf{Total amount of characters}}
\end{equation}

For the compression system, we also need a code alphabet with which we want to compress our code, the simplest of which being the binary system with the code alphabet $\mathcal{A}=\{0,1\}$ where $r=2$. This tells us that each internal node in the tree has two children and only one sibling. For most of the examples, we will use this code alphabet for simplicity, but you should keep in mind that all of these systems can also work with larger code alphabets.
\begin{equation}
	\mathcal{A}=\{a_1 \to a_r\}
\end{equation}

The compression tree with which the message code is encoded and decoded is depicted with $\mathcal{C}$. Each of the nodes in the compression tree is listed in the array $\mathcal{Q}$ with $n$ being the total amount of nodes in the compression tree. 
\begin{equation}
	\mathcal{Q}=\{q_1 \to q_n\}
\end{equation}

Each node $q_i$ in the compression tree has a corresponding weight $\tau_i$, which can be calculated as the sum of the weights of the children $j$ of that node and is kept in an array $\mathcal{T}$. 
\begin{equation}
	\mathcal{T}=\{\tau_1 \to \tau_i\}
\end{equation}
\begin{equation}
	\tau_{i}=\sum^j_{o=1}{\tau_o}
\end{equation}

After the encoding process using the compression tree $\mathcal{C}$, the result is an output sequence $\mathcal{Y}$ that is a concatenation of all the resulting code words with a length $L_\mathcal{Y}$. In addition, we will use the array $\mathcal{X}$ to list the resulting code words.
\begin{equation}
	\mathcal{X}=\{x_1 \to x_m\}
\end{equation}

The resulting encoding scheme that acts like a translation table between encoded and decoded codes is listed in $\phi$.
\begin{equation}
	\phi:s_1\to x_1,\dots, s_i \to x_i
\end{equation}

The average length of the code words in $\mathcal{X}$ is represented by $\overline{L}$.
\begin{equation}
	\overline{L}(\mathcal{X})=\sum^m_{i=1}L(x_i)
\end{equation}

The main objective of our compression systems is to assign the most frequent input sequence character $s_1$ to the shortest code word $x_1$, the ideal being that the average code word length $\overline{L}$ is as small as possible. This smallest possible average code word length is referred to as a minimum redundancy code, coined by David A. Huffman in his paper on the construction of just such a code under the conditions that: No two messages shall have the same resulting sequence $\mathcal{Y}$ and each code word shall be distinguishable without needing to specify the start or end of a code word.\footnote{\cite{huffmanOriginal} p.1098: Introduction}
	\subsection{Huffman Coding}
In Huffman's system, we start with a sequence code $\mathcal{S}$ that is sorted by descending probability and assign each of these characters a node in $\mathcal{Q}$.
\begin{equation}
	p_i\geq p_{i+1}\geq \dots \geq p_m
\end{equation}
\begin{equation}
	q_i \leftarrow s_i, (\tau_i= p_i)
\end{equation}

We then take a maximum of $r$ but at least 2 characters with the lowest probability and connect them together with a new node and put this new node back in the array $\mathcal{Q}$ in the right position corresponding to its weight.
\begin{equation}
	q_{new}\leftarrow {q_{new}}_j, \tau_{new}=\sum^j_{i=1}\tau_i
\end{equation}

You can repeat this process until $\tau_{new}=1$. At the end, we can visualise the compression tree by tracing out the children of each node until we reach the source characters. \footnote{\cite{huffmanOriginal} p.1011: Generalization of the Method; \cite{rueda2002advances} p.50: Algorithm 1 Static Huffman Encoding}

\subsubsection{Example: Static Huffman Coding}

Input:
\begin{itemize}
	\item $\mathcal{S}=\{a,b,c,d,e,f,g,h\}$
	\item $\mathcal{P}=\{0.22,0.20,0.18,0.15,0.10,0.08,0.05,0.02\}$
	\item $\mathcal{A}=\{0,1,2,3\}$
\end{itemize}

\begin{table}[ht]
	\centering
	\begin{tabular}{cccc}
		$s_i$ & $p_i$ & $L_i$ & Code \\
		a & 0.22 & 1 & 1 \\
		b & 0.20 & 1 & 2 \\
		c & 0.18 & 1 & 3 \\
		d & 0.15 & 2 & 00 \\
		e & 0.10 & 2 & 01 \\
		f & 0.08 & 2 & 02 \\
		g & 0.05 & 3 & 030 \\
		h & 0.02 & 3 & 031 \\
	\end{tabular}
	\caption{Adapted from \cite{huffmanOriginal} p.1101: Table III}
	\label{tab:placeholder}
\end{table}

\begin{figure}[ht]
	\centering
	\includegraphics[width=1.1\linewidth]{huffmanTree}
	\caption{Huffman Tree of Table 1; Made using Microsoft PowerPoint}
	\label{fig:huffmantree}
\end{figure}

In Figure 1, the green nodes represent the single characters with their corresponding probabilities $p$ and the purple nodes are the connecting nodes $q$ with their corresponding weight $\tau$. While encoding and decoding, we use this tree as a translation guide, the furthest left being 0, then 1, and so on.


\subsubsection{Adaptive Huffman Coding}
Adaptive Huffman Coding or Dynamic Huffman Coding, in comparison to the static version, is a system that adapts the construction of the Huffman tree to changes in the count of characters, rather than having a presorted list with given probabilities. This theory suggested by Robert G. Gallager, is based on the sibling property, where if the nodes can be listed in decreasing counts and each node is adjacent to its sibling, then the code is optimal for the current character count\footnote{\cite{gallager2003variations} p.6: 2. The Sibling Property Definition}. In a binary system each node, or sibling pair, contains 5 different components. Two are the current counters of the children nodes, two are links to the children nodes, and the last is a link to parent node. Each of the links also having an extra property indicating whether the node is the 0th or the 1st child node. As an input to our adaptive system, we have the Huffman tree of one less character read and the next character. The output then being the Huffman tree with the updated count. With the incrementing of the character count, if it is found that the count exceeds the count of the next higher sibling pair, then these two nodes are interchanged, changing the code and the resulting Huffman tree\footnote{\cite{gallager2003variations} p.17-21: Adaptive Huffman Codes}. This algorithm was then further expanded upon by Donald E. Knuth with his paper on Dynamic Huffman Coding. Here he fills in some gaps left by Gallager's adaptive system like the decrementing of the character counts\footnote{\cite{knuthDynamicHuffman} p.163-165: Introduction}.
	\subsection{Shannon Coding}
Even though the Shannon Method suggested by Claude E. Shannon is not commonly used due to its inefficiency when it comes to the resulting compression size, it still has a relevance in the field. Compared to the Huffman Coding system, this one is based on a top-down construction system making it more intuitive and faster to calculate. This system gives a unique string of numbers by using a cumulative probability function $P_i$ and then taking the first $l_i$ digits of the $r$th decimal representation of the cumulative probability function to create the corresponding code word\footnote{\cite{rueda2002advances} p.52-53: 2.4.2 Shannon's Method; \cite{shannonMathematicalCommunication} p.58-61: 9. The Fundamental Theorem for a Noiseless Channel}.
\begin{equation}
	P_i = \left\{
	\begin{array}{lll}
		0 & \textbf{for} & i = 1 \\
		P_{i-1} + p_{i-1} & \textbf{for} & 2 \leq i \leq m
	\end{array}
	\right.
\end{equation}


\begin{equation}
	l_i=\lceil log_r(p_i^{-1})\rceil
\end{equation}

\subsubsection{Example: Static Shannon Coding}

\todo{Make example}
\todo{Make example encoding scheme using example}


\subsubsection{Adaptive Shannon Coding}

Travis Gagie suggests a method of making an adaptive Shannon coding system. In their method, we separate tasks into foreground and background tasks. The foreground task takes the weight of the previous character counts and updates the weights to the new input character. If the character isn't in the encoding scheme yet, the node is added into the character count. In the background, for each character processed in the foreground, a character is processed in the background. The weight of a certain character $a$ is then updated if the character has occured before, or it is added to the compression tree with the weight corresponding to the characters probability\footnote{\cite{gagie2004dynamic} p.3-4: III Dynamic Shannon Coding}. This system is similar to the Brute-Force adaptive Fano coding system discussed later in this paper. 

\todo{Explanation of difference between static}
\todo{Make example}
	\subsection{Fano Coding}
The Fano coding method is, like the Shannon coding method, based on a top-down construction system. The Fano method starts with a sequence code $\mathcal{S}$ that is sorted by descending probability in $\mathcal{P}$ (See Equation 11). In the binary system, we then split this list into two subgroups of equal probability $\mathcal{S}_0$ and $\mathcal{S}_1$, and as a result also $\mathcal{P}_0$ and $\mathcal{P}_1$. Their codes then starting with either 0 or 1. This process is then repeated until the subdivisions are equal to the single characters. Here the code word length is equal to $\log_2 m$ if $m$ is a power of two. If $m$ is not a power of two, the length of the code word is one of the two closest integers to $\log_2 m$.\footnote{\cite{fanoTransmissionInformation} p.5-6: Selection from N Equally Likely Choices; \cite{rueda2002advances} p.55: Algorithm 3 Static Fano Encoding}

\todo{Exact explanation of process}
\todo{Make Example}
\todo{Make example tree using example}


\subsubsection{Adaptive Fano Coding}

The adaptive Fano coding system functions using the same concept as the adaptive Huffman coding system in the sense that the source character weights are not calculated beforehand. Luis Rueda suggests two different ways to make an adaptive Fano coding system, one they call the brute-force method and the other the greedy method. In the brute-force method for each change in the character counter, the system recomputes the corresponding compression tree. This is of course rather inefficient\footnote{\cite{rueda2006fast} p.1659-1660: 2.1 A brute-force method for adaptive Fano coding}. In the greedy method, we start with a list of the input alphabet and initialise the probabilities of each character to be equal. Whenever a character is encoded a specific partitioning procedure is invoked that outputs the code word. Depending on the code alphabet used, different partitioning procedures can be used, for the sake of simplicity, we will just say that this system in some way or another partitions the probabilities optimally but more information into different partitioning procedures can be found in \cite{rueda2006fast}. As a result however when encoding a character in our input sequence we no longer need to update our compression tree for every new input character. Only the changes in weight of a specific character are imputed and any required changes in the current encoding scheme\footnote{\cite{rueda2006fast} p.1660: 2.2 The greedy encoding algorithm}.

\todo{Explain difference to static}
\todo{Make example}
	\subsection{Further Improvements}
\subsubsection{Lempel-Ziv Algorithm}

\todo{Rough Explanation of LZ77 and LZ78}

\subsubsection{Deflate Algorithm}

The Deflate Algorithm is a higher-order system based on both the LZ77 system and the Huffman coding system. 

\todo{Explanation of combination of Huffman and LZ}
\todo{Use cases, improvements}

\subsubsection{Lempel-Ziv-Welch Algorithm}

\todo{General explanation reference to introduction explanation}
	\subsection{Comparison System}
The way that we have compared the different compression systems is a program in which we compare the amount of calculations used and the size of the resulting compressed file relative to the original file size. 
	\subsubsection{Surrounding System}
\todo{Insert into compression code}
\todo{Return compressed size, number of calculations}

Processes that do not vary between the different compression systems will not be included in the resulting calculation time. The metric of calculation time is supposed to be a comparison between the construction of the compression tree. Therefore, the static compression system that are based on an already constructed compression tree, do not show the metric of the number of calculation processes. 
	\subsection{Results}

\begin{table}[h]
	\centering
	\begin{tabular}{cccc}
		\textbf{Data} & \textbf{ASCII-1965\tablefootnote{\cite{mackenzie1980coded} p.423-425: 23.2 ASCII-1965; also ISO/IEC 646: \cite{iso646ascii7bit} p.6-7: Table 1 (Calculated: Number of Characters multiplied by 7)}} & \textbf{USASCII-8\tablefootnote{\cite{mackenzie1980coded} p.431-433: 23.11 USASCII-8 (Calculated: Number of Characters multiplied by 8)}} & \textbf{UTF-32\tablefootnote{\cite{unicode170} p.77: 2.5.1 UTF-32 (Calculated: Number of Characters multiplied by 32)}} \\
		Wordnik\tablefootnote{\cite{wordnikWords}: Wordnik Public Domain List of English Words} & $16.57$ & $18.93$ & $75.74$ \\
		Shakespeare\tablefootnote{\cite{shakespeareCompleteWorks}: The Complete Works of William Shakespeare} & -\tablefootnote{File from Project Gutenberg contains some formatting characters that aren't included in the standard ASCII-1965} & $43.03$ & $172.12$ \\
	\end{tabular}
	\caption{List of Examples used for comparing the different compression systems with their corresponding storage usage in Mb.}
	\label{tab:placeholder}
\end{table}

\begin{table}[h]
	\centering
	\begin{tabular}{c|c|c|c|c}
		\textbf{Compression System} & \textbf{Processes\tablefootnote{Calculation processes are relative, different types of processes take a different amount of time so this number should be interpreted rather vaguely.}} & \textbf{Size\tablefootnote{Storage usage after compression not including the storage used by the compression tree or encoding scheme. The storage taken by these in proportion to a larger amount of data makes it so that we can basically ignore this small extra storage usage.}} & \textbf{$\overline{L}$ Avg. L\tablefootnote{Average Code Word Length (See Section 3: Equation 10)}} & \textbf{Percentage Saved\tablefootnote{Percentage of the compressed storage size compared to the smallest working coded character set in Table 2.}} \\
		Static Huffman & $-$ & $-$ & $~-$ & $~-$ \\
		Static Shannon & $-$ & $-$ & $~-$ & $~-$ \\
		Static Fano & $-$ & $-$ & $~-$ & $~-$ \\
	\end{tabular}
	\caption{Comparison of the different compression systems with Wordnik}
	\label{tab:placeholder}
\end{table}

\begin{table}[h]
	\centering
	\begin{tabular}{c|c|c|c|c}
		\textbf{Compression System} & \textbf{Processes} & \textbf{Size} & \textbf{$\overline{L}$ Avg. L} & \textbf{Percentage Saved} \\
		Static Huffman & $-$ & $-$ & $~-$ & $~-$ \\
		Static Shannon & $-$ & $-$ & $~-$ & $~-$ \\
		Static Fano & $-$ & $-$ & $~-$ & $~-$ \\
	\end{tabular}
	\caption{Comparison of the different compression systems with Shakespeare}
	\label{tab:placeholder}
\end{table}
	\section{Conclusion}
In general, even though the original compression systems of Huffman, Shannon, and Fano are not used that frequently any more in their original forms, they lay the foundation of modern lossless compression systems. Huffman shows us that in a zeroth order compression system, an optimal compression algorithm exists and functions using a bottom-up construction system. Fano and Shannon coding system on the other hand might not be as effective when it comes to getting close to the value of the entropy, their coding systems can give a more intuitive overview and way of implementing a lossless compression algorithm. Looking at the examples we used we can see though, that the difference between their compression sizes is quite small at first but increases the larger the file size gets. In the example we can also see Shannon's theory work, where the larger the file size gets the more storage is saved. When we start looking at higher-order models, things can get much more complicated and more storage efficient, but it almost always ends up with a zeroth-order compression system just that the input can be more effectively compressed than the original input. 


	
	\newpage
	\printbibliography
	
\end{document}
