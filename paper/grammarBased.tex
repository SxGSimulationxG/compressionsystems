\subsection{Grammatikbasierte Codierung}

Bei der grammatikbasierten Codierung (englisch: \textit{Grammar-Based Coding}) wird die Eingabesequenz nicht direkt codiert, sondern über eine Grammatik beschrieben, die genau diese Sequenz erzeugt. Dadurch lassen sich Wiederholungen und strukturelle Muster in der Zeichenkette effizient darstellen\footnote{\cite{rueda2002advances} p.80–81: 2.6.6 Grammar-Based Coding}.

Eine Grammatik $G$ besteht aus:
\begin{itemize}
	\item einer Menge von Terminalsymbolen $T$ (oder Alphabet $A$ oder $\Sigma$),
	\item einer Menge von Nichtterminalen $N$,
	\item einer Menge von Produktionsregeln $P$,
	\item und einem Startsymbol $s$.
\end{itemize}

Mit diesem höheren Abstraktionsgrad wird nicht die Zeichenkette $x$ selbst codiert, sondern die Grammatik $G_x$, die $x$ eindeutig erzeugt.  
Da $G_x$ genau eine Zeichenkette erzeugt, handelt es sich um eine kontextfreie Grammatik, für die gilt:
\begin{equation}
	L(G_x) = \{x\}
\end{equation}

Das Ziel der grammatikbasierten Codierung besteht darin, die kleinstmögliche Grammatik zu finden, die eine gegebene Zeichenfolge $x$ beschreibt.  
Die Größe einer Grammatik $|G_x|$ entspricht der Anzahl der Symbole auf der rechten Seite der Produktionsregeln $P$.  
Dieses Optimierungsproblem wird als Smallest Grammar Problem (SGP) bezeichnet\footnote{\cite{charikar2005smallest} p.1–2: Introduction}.

Ein Beispiel aus \cite{charikar2005smallest} verdeutlicht das Prinzip:
\begin{equation}
	P = \{
	\langle S \rangle \to \langle B \rangle \langle B \rangle \langle A \rangle,\quad
	\langle A \rangle \to \text{a rose},\quad
	\langle B \rangle \to \langle A \rangle \text{ is}
	\}
\end{equation}

Diese Grammatik erzeugt die Zeichenkette „a rose is a rose is a rose“ und besitzt die minimale Größe $|G_x| = 14$.  
Damit beschreibt sie die Struktur der Wiederholungen sehr effizient, da gleiche Teilsequenzen nur einmal definiert werden müssen.

Das Finden der kleinsten Grammatik ist ein NP-schweres Problem.  
Ein Problem wird als NP-schwer bezeichnet, wenn es mindestens so schwierig ist wie die schwierigsten Probleme in der Komplexitätsklasse NP (nichtdeterministisch polynomiell).  
Das bedeutet, dass bisher kein Algorithmus bekannt ist, der das Problem in polynomieller Zeit für alle Eingaben lösen kann.  
Stattdessen wächst der Rechenaufwand mit zunehmender Eingabelänge exponentiell.  
In der Praxis greift man daher auf Heuristiken oder Approximationsverfahren zurück, um eine hinreichend kleine, aber nicht unbedingt optimale Grammatik effizient zu finden.
\todo{citation needed}