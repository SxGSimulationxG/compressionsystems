\subsection{Dictionary Based Kodierung}
Es gibt viele Beispiele für Dictionary Based Kodierung, einer davon von Storer und Szymanski in ihrer Arbeit „Data Compression via Textual Substitution”, wo sie eine mögliche System vorschlagen. Ihr System füht mehrere Komponenten der normale Kompression hinzu: externe Makroschemata, interne Makroschemata, ein Wörterbuch (dictionary) und ein Skelett. Das externe Makroschema ermöglicht die Kodierung einer Quellzeichenfolge unter Verwendung eines Wörterbuchs, eines Speichers für Referenzzeichenfolgen und eines Skeletts, einer Kombination aus Zeichen des Eingabealphabets und Zeigern auf das Wörterbuch. Die internen Makro-Schemata ermöglichen Zeigern auf wiederholte Abschnitte derselben Zeichenfolge\footnote{\cite{storer1982data} p.929-932: 2 The Model and Basic Definitions}. Mit diesem System können wir den Grad der Redundanz in unserer Eingabesequenz verringern. Dieses Algorithmus, auch LZSS genannt, wurde in den 1980er Jahre für Archivierungssoftware verwendet, wie das ARJ oder RAR Format\footnote{\cite{steinruecken2015lossless} p.30-31: 2.3.2 History and Significance}. Weitere Beispiele für Dictionary Based Kodierungssystemen wären LZ77, LZW, DEFLATE und LZMA. 

\todo{LZ77/Deflate}