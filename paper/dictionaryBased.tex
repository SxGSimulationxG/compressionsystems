\subsection{Dictionary Based Kodierung}
Es gibt viele Beispiele für die Dictionary Based Kodierung, eines davon stammt von Storer und Szymanski in ihrer Arbeit „Data Compression via Textual Substitution“, in der sie ein allgemeines Modell für Datenkompression vorstellen. Ihr System erweitert die herkömmliche Kompression um mehrere Komponenten: externe Makroschemata, interne Makroschemata, ein Wörterbuch (dictionary) und ein Skelett.
Das externe Makroschema ermöglicht die Kodierung einer Quellzeichenfolge mithilfe eines Wörterbuchs, eines Speichers für Referenzzeichenfolgen und eines Skeletts – einer Kombination aus Zeichen des Eingabealphabets und Zeigern auf das Wörterbuch. Das interne Makroschema hingegen erlaubt Zeiger auf wiederholte Abschnitte innerhalb derselben Zeichenfolge\footnote{\cite{storer1982data} p.929-932: 2 The Model and Basic Definitions}. Durch diese Technik wird die Redundanz in der Eingabesequenz verringert. Das in Data Compression via Textual Substitution vorgestellte Makromodell bildet die theoretische Grundlage für Verfahren wie LZ77 und dessen Varianten (z. B. LZSS), die später in Programmen wie ARJ oder RAR eingesetzt wurden.. Weitere Beispiele für Dictionary Based Kodierungssystemen wären LZ77, LZW, DEFLATE und LZMA\footnote{\cite{steinruecken2015lossless} p.30-31: 2.3.2 History and Significance}. 