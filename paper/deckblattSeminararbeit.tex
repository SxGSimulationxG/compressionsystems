% Deckblatt für Seminararbeit (inputfähig)
% Eingebunden mit: \input{deckblatt_seminararbeit_input.tex}

% --- Lokale Einstellungen ---

\newcommand{\Fline}[1][8cm]{\rule{#1}{0.4pt}}
\newcommand{\Label}[1]{\textbf{#1}}

% --- Variablen (vom Hauptdokument befüllbar) ---
\providecommand{\Schule}{Friedrich-Koenig-Gymnasium Würzburg}
\providecommand{\Jahrgang}{Oberstufenjahrgang 2024/2026}
\providecommand{\Seminartyp}{W-Seminar}
\providecommand{\Fach}{Mathematik}
\providecommand{\Thema}{Quellenkodierung I - Huffman-/Shannon-Fano-Kodierung}
\providecommand{\Verfasser}{Kai Rasmussen}
\providecommand{\Kursleiter}{Oliver Manger}
\providecommand{\SpaetestAbgabe}{Di., 11.11.2025}
\providecommand{\AbgabeTagOS}{}
\providecommand{\OrtDatum}{}
\providecommand{\Seminar}{}
\providecommand{\NameVorname}{}
\providecommand{\KursleiterAmt}{}

% --- Deckblatt ---
\begin{center}
	{\textbf{\Schule}}
	{\textbf{\Jahrgang}}\\[14mm]
	{\Large \textbf{SEMINARARBEIT}}\\[4mm]
	{\large \underline{im \Seminartyp}}\\[5mm]
	{\Large WM Kryptologie}\\[5mm]
	\textbf{im Fach}\\[2mm]
	Mathematik
\end{center}

\Label{Thema:}
\begin{center}
	\large \Thema\\[8mm]
\end{center}


\begin{tabularx}{\textwidth}{@{}>{\raggedright\arraybackslash}p{50mm}X@{}}
	\Label{Verfasser / -in:} & \Verfasser\\[4mm]
	\Label{Kursleiter / -in:} & \Kursleiter\\[8mm]
	\Label{spät. Abgabetermin:} & \SpaetestAbgabe\\[4mm]
	\Label{Tag der Abgabe im OS-Büro:} & \Fline[3cm]\\
\end{tabularx}

\vfill

\begin{tcolorbox}[colback=white,colframe=black,width=\textwidth,
	boxrule=0.5pt,arc=1mm,outer arc=0pt, left=5mm, right=5mm, top=2mm, bottom=2mm]
	\renewcommand{\arraystretch}{1.8}
	
	\begin{tabular}{@{}>{\bfseries}ll@{}}
		Bewertung: &  \fbox{\rule{2cm}{0pt}\rule[-3pt]{0pt}{12pt}} \ Punkte für die W-Seminararbeit \\
		& \small (einfache Wertung, Ergebnis liegt zwischen 0 und 15 Punkten) \\[1ex]
		& \fbox{\rule{2cm}{0pt}\rule[-3pt]{0pt}{12pt}} \ Punkte für die Präsentation \\[3ex]
		Gesamtnote: &
		\fbox{\rule{2cm}{0pt}\rule[-3pt]{0pt}{12pt}} \ Punkte \\[0.5ex]
		& \small (Arbeit dreifach gewertet, Präsentation einfach gewertet, Summe geteilt\\ 
		& \small durch 2, Ergebnis gerundet; Gesamtergebnis liegt zwischen\\ 
		& \small 0 und 30 Punkten)
	\end{tabular}
\end{tcolorbox}

\vfill

\begin{flushleft}	
	Besprochen am: \Fline[4cm]\\[10mm]
	\hfill \Fline[10cm]\\[-1mm]
	\hfill \small Unterschrift der Kursleiterin / des Kursleiters
\end{flushleft}