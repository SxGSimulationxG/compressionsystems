\section{Grundlegende Kompressionssysteme}
\subsection{Shannon Codierung}
Auch wenn die von Claude E. Shannon vorgeschlagene Shannon-Methode aufgrund ihrer Ineffizienz hinsichtlich der resultierenden Kompressionsgröße nicht häufig verwendet wird, hat sie dennoch eine gewisse Relevanz in diesem Bereich. Dieses System liefert eine eindeutige Zahlenfolge, indem es eine kumulative Wahrscheinlichkeitsfunktion $P_i$ verwendet und dann die ersten $l_i$ Ziffern der $r$-ten Dezimaldarstellung der kumulativen Wahrscheinlichkeitsfunktion nimmt, um das entsprechende Codewort zu erstellen\footnote{\cite{rueda2002advances} p.52-53: 2.4.2 Shannon's Method; \cite{shannonMathematicalCommunication} p.401-403: 9. The Fundamental Theorem for a Noiseless Channel}.

\begin{equation}
	P_i = \left\{
	\begin{array}{lll}
		0 & \textbf{for} & i = 1 \\
		P_{i-1} + p_{i-1} & \textbf{for} & 2 \leq i \leq m
	\end{array}
	\right.
\end{equation}


\begin{equation}
	l_i=\lceil log_r(p_i^{-1})\rceil
\end{equation}

Shannon stellt außerdem fest, dass mit zunehmender Länge $N$ unserer Eingabesequenz die durchschnittliche Länge des Codeworts $H'$ sich der Entropie $H$ annähert.

\begin{equation}
	H'=\frac{1}{N} \sum(m_s p_s)
\end{equation}

Hier ist $p_s$ der Wert für die kumulative Wahrscheinlichkeitsfunktion $P_i$. $m_s$ ist eine ganze Zahl, wobei:
\begin{equation}
	\log_2(\frac{1}{p_s}) \leq m_s < \log_2(\frac{1}{p_s}) + 1
\end{equation}

\begin{equation}
	H \leq H' < H + \frac{1}{N}
\end{equation}

Wenn diese Annahme richtig ist, bestätigt dies die Intuition, dass ein Komprimierungssystem bei längeren Eingaben umso effektiver ist, auch wenn das Komprimierungssystem nicht optimal ist. Der Wert für die durchschnittliche Codewortlänge kann jedoch niemals unter den Wert für die Entropie $H$ fallen, ohne dass Informationen verloren gehen.

\subsubsection{Beispiel: Statisches Shannon Kodierung}

Eingabe:
\begin{itemize}
	\item $\mathcal{S}=\{a,b,c,d,e,f,g,h\}$
	\item $\mathcal{P}=\{0.22,0.20,0.18,0.15,0.10,0.08,0.05,0.02\}$
	\item $\mathcal{A}=\{0,1\} \to r=2$
\end{itemize}

\begin{table}[ht]
	\centering
	\begin{tabular}{c|c|c|c|c|c}
		$s_i$ & $p_i$ & $P_i$ & $l_i$ & $r$-Darstellung von $P_i$ & Code \\
		a & 0.22 & 0 & 3 & $0.000\dots$ & 000 \\
		b & 0.20 & 0.22 & 3 & $0.001\dots$ & 001 \\
		c & 0.18 & 0.42 & 3 & $0.011\dots$ & 011 \\
		d & 0.15 & 0.60 & 3 & $0.100\dots$ & 100 \\
		e & 0.10 & 0.75 & 4 & $0.11$ & 1100 \\
		f & 0.08 & 0.85 & 4 & $0.1101\dots$ & 1101 \\
		g & 0.05 & 0.93 & 5 & $0.11101\dots$ & 11101 \\
		h & 0.02 & 0.98 & 6 & $0.111110\dots$ & 111110 \\
	\end{tabular}
	\caption{Angepasst aus Beispiel von \cite{huffmanOriginal} p.1101: Table III}
	\label{tab:placeholder}
\end{table}

\todo{encoding scheme}

\subsubsection{Adaptive Shannon Kodierung}

Travis Gagie schlägt eine Methode zur Erstellung eines adaptiven Shannon-Kodierungssystems vor. Bei dieser Methode werden die Aufgaben in Vordergrund- und Hintergrundaufgaben unterteilt. Die Vordergrundaufgabe berücksichtigt das Gewicht der vorherigen Zeichenanzahl und aktualisiert die Gewichte entsprechend dem neu eingegebenen Zeichen. Befindet sich das Zeichen noch nicht im Kodierungsschema, wird der Knoten zur Zeichenanzahl hinzugefügt. Im Hintergrund wird für jedes im Vordergrund verarbeitete Zeichen ein Zeichen im Hintergrund verarbeitet. Die Gewichtung eines bestimmten Zeichens $a$ wird dann aktualisiert, wenn das Zeichen zuvor aufgetreten ist, oder es wird mit der Gewichtung, die der Wahrscheinlichkeit des Zeichens entspricht, zum Komprimierungsbaum hinzugefügt\footnote{\cite{gagie2004dynamic} p.3-4: III Dynamic Shannon Coding}. Dieses System ähnelt dem adaptiven Fano-Kodierungssystem mit der Brute-Force-Methode, das später behandelt wird. 
\newpage
\subsection{Fano Codierung}
Die Fano-Kodierungsmethode basiert wie die Shannon-Kodierungsmethode auf einem Top-Down-Konstruktionssystem. Die Fano-Methode beginnt mit einem Sequenzcode $\mathcal{S}$, der nach absteigender Wahrscheinlichkeit in $\mathcal{P}$ sortiert ist. Im binären System teilen wir diese Liste dann in zwei Untergruppen mit ungefähr gleicher Wahrscheinlichkeit $\mathcal{S}_0$ und $\mathcal{S}_1$ und damit auch $\mathcal{P}_0$ und $\mathcal{P}_1$. Ihre Codes beginnen dann entweder mit 0 oder 1. Dieser Vorgang wird dann so lange wiederholt, bis die Unterteilungen den einzelnen Zeichen entsprechen. Hier ist die Codewortlänge gleich $\log_2 m$, wenn $m$ eine Potenz von zwei ist. Wenn $m$ keine Zweierpotenz ist, ist die Länge des Codeworts eine der beiden ganzen Zahlen, die $\log_2 m$ am nächsten kommen\footnote{\cite{fanoTransmissionInformation} p.5-6: Selection from N Equally Likely Choices; \cite{rueda2002advances} p.55: Algorithm 3 Static Fano Encoding}.

\subsubsection{Beispiel: Statisches Fano Kodierung}
\begin{itemize}
	\item $\mathcal{S}=\{a,b,c,d,e,f,g,h\}$
	\item $\mathcal{P}=\{0.22,0.20,0.18,0.15,0.10,0.08,0.05,0.02\}$
	\item $\mathcal{A}=\{0,1\} \to r=2$
\end{itemize}

\begin{table}[ht]
	\centering
	\begin{tabular}{c|c|c|c}
		$s_i$ & $p_i$ & $L_i$ & Code \\
		a & 0.22 & 2 & 00 \\
		b & 0.20 & 2 & 01 \\
		c & 0.18 & 3 & 100 \\
		d & 0.15 & 3 & 101 \\
		e & 0.10 & 3 & 110 \\
		f & 0.08 & 4 & 1110 \\
		g & 0.05 & 5 & 11110 \\
		h & 0.02 & 5 & 11111 \\
	\end{tabular}
	\caption{Angepasst aus Beispiel von \cite{huffmanOriginal} p.1101: Table III}
	\label{tab:placeholder}
\end{table}

\begin{figure}[ht]
	\centering
	\includegraphics[width=1.1\linewidth]{fanoTree}
	\caption{Fano Baum von Tabelle 2; Mit Manim Community v0.19.0 gemacht}
	\label{fig:fanotree}
\end{figure}

\newpage

\subsubsection{Adaptive Fano Codierung}
Luis Rueda schlägt zwei verschiedene Möglichkeiten vor, um ein adaptives Fano-Kodierungssystem zu erstellen: die sogenannte Brute-Force-Methode und die sogenannte Greedy-Methode. Bei der Brute-Force-Methode berechnet das System bei jeder Änderung des Zeichenzählers den entsprechenden Kompressionsbaum neu\footnote{\cite{rueda2006fast} p.1659-1660: 2.1 A brute-force method for adaptive Fano coding}. Dies ist natürlich ziemlich ineffizient, aber funktioniert immer um einen statisches in ein adaptives Kompressionssystem zu wandeln. 
Bei der Greedy-Methode beginnen wir mit einer Liste des Eingabealphabets und initialisieren die Wahrscheinlichkeiten jedes Zeichens als gleich. Immer wenn ein Zeichen codiert wird, wird ein bestimmtes Partitionierungsverfahren aufgerufen, das das Codewort ausgibt. Je nach verwendetem Codealphabet können verschiedene Partitionierungsverfahren zum Einsatz kommen. Der Einfachheit halber sagen wir einfach, dass dieses System die Wahrscheinlichkeiten auf die eine oder andere Weise optimal partitioniert, aber weitere Informationen zu verschiedenen Partitionierungsverfahren finden Sie in \cite{rueda2006fast}. Das Ergebnis ist jedoch, dass wir bei der Codierung eines Zeichens in unserer Eingabesequenz nicht mehr unseren Komprimierungsbaum für jedes neue Eingabezeichen aktualisieren müssen. Es werden nur die Änderungen im Gewicht eines bestimmten Zeichens berücksichtigt und alle erforderlichen Änderungen im aktuellen Kodierungsschema\footnote{\cite{rueda2006fast} p.1660: 2.2 The greedy encoding algorithm}.

\todo{Beispiel}
\newpage
\subsection{Huffman Codierung}
Die Huffman-Kodierung ist ein Verfahren zur Erstellung eines optimalen Präfixcodes für ein gegebenes Sequenzcode mit bekannten Wahrscheinlichkeiten. Das Grundprinzip besteht darin, dass häufig auftretende Symbole kürzere und seltene längere Codewörter erhalten. Zu Beginn wird die Sequenzcode $\mathcal{S}$ nach absteigender Wahrscheinlichkeit sortiert. Jedes Symbol $s_i$ wird durch einen Knoten $q_i$ mit Gewicht $\tau_i = p_i$ in der Liste $\mathcal{Q}$ dargestellt. 

\begin{equation}
	p_i\geq p_{i+1}\geq \dots \geq p_m
\end{equation}
\begin{equation}
	q_i \leftarrow s_i, (\tau_i= p_i)
\end{equation}

Wir nehmen dann maximal $r$, mindestens jedoch 2 Zeichen mit der geringsten Wahrscheinlichkeit, verbinden sie mit einem neuen Knoten und fügen diesen neuen Knoten wieder in die richtige Position entsprechend seinem Gewicht in das Array $\mathcal{Q}$ ein. Durch das wiederholte Zusammenfassen der unwahrscheinlichsten Knoten entstehen in der Baumstruktur kurze Wege für häufige Symbole und lange Wege für seltene. Dadurch wird die Gesamtlänge aller Codewörter minimiert. 

\begin{equation}
	q_{new}\leftarrow {q_{new}}_j, \tau_{new}=\sum \tau_i
\end{equation}

Dieser Vorgang wird so lange wiederholt, bis der gesamte Wahrscheinlichkeitsraum durch einen einzigen Wurzelknoten mit $\tau_{new}=1$ dargestellt wird. Am Ende kann man den Kompressionsbaum visualisieren, indem man die Kinder jedes Knotens nachverfolgen, bis man die Quellzeichen erreicht\footnote{\cite{huffmanOriginal} S.1011: Generalization of the Method; \cite{rueda2002advances} S.50: Algorithm 1 Static Huffman Encoding}.

\subsubsection{Beispiel: Statisches Huffman Codierung}

Eingabe:
\begin{itemize}
	\item $\mathcal{S}=\{a,b,c,d,e,f,g,h\}$
	\item $\mathcal{P}=\{0.22,0.20,0.18,0.15,0.10,0.08,0.05,0.02\}$
	\item $\mathcal{A}=\{0,1,2,3\}$
\end{itemize}

\begin{table}[ht]
	\centering
	\begin{tabular}{cccc}
		$s_i$ & $p_i$ & $L_i$ & Code \\
		a & 0.22 & 1 & 1 \\
		b & 0.20 & 1 & 2 \\
		c & 0.18 & 1 & 3 \\
		d & 0.15 & 2 & 00 \\
		e & 0.10 & 2 & 01 \\
		f & 0.08 & 2 & 02 \\
		g & 0.05 & 3 & 030 \\
		h & 0.02 & 3 & 031 \\
	\end{tabular}
	\caption{Beispiel von \cite{huffmanOriginal} S.1101: Table III}
	\label{tab:placeholder}
\end{table}

\begin{figure}[ht]
	\centering
	\includegraphics[width=1.1\linewidth]{huffmanTree}
	\caption{Huffman Baum von Tabelle 3; Mit Manim Community v0.19.0 erstellt}
	\label{fig:huffmantree}
\end{figure}

In Abbildung 2 stellen die grünen Knoten die einzelnen Zeichen mit ihren entsprechenden Wahrscheinlichkeiten $p$ dar, während die violetten Knoten die Verbindungsknoten $q$ mit ihrem entsprechenden Gewicht $\tau$ darstellen. Bei der Kodierung und Dekodierung verwenden wir diesen Baum als Übersetzungshilfe, wobei der am weitesten links stehende Knoten 0 ist, dann 1 usw.

\newpage

\subsubsection{Adaptive Huffman Codierung}
Die adaptive oder dynamische Huffman-Kodierung erweitert das klassische statische Verfahren um die Fähigkeit, sich während der Kodierung automatisch an Veränderungen der Symbolhäufigkeiten anzupassen. Im Gegensatz zur statischen Variante, bei der die Wahrscheinlichkeiten der Symbole im Voraus bekannt sein müssen, wird der Huffman-Baum hier fortlaufend angepasst, während die Eingabedaten gelesen werden. Dadurch kann die Kompression in einem einzigen Durchlauf erfolgen, ohne dass ein vorheriger Statistikaufbau erforderlich ist. Die Grundidee besteht darin, dass der Baum nach jedem neu eingelesenen Zeichen so umstrukturiert wird, dass er weiterhin die optimalen Eigenschaften der Huffman-Kodierung erfüllt. Diese Eigenschaft wird durch die sogenannte Geschwister-Eigenschaft (engl. sibling property) definiert. Sie besagt, dass für eine gegebene Kodierung der Baum genau dann optimal ist, wenn alle Knoten in absteigender Reihenfolge ihrer Gewichtungen angeordnet werden können und jedes Knotenpaar von Geschwistern aufeinanderfolgende Positionen in dieser Ordnung einnimmt\footnote{\cite{gallager2003variations} S.6: 2. The Sibling Property Definition}. In einem binären System besteht jeder Knoten (oder jedes Geschwisterpaar) aus fünf Komponenten: zwei Zählern für die untergeordneten Knoten, zwei Zeigern auf diese untergeordneten Knoten sowie einem Zeiger auf den übergeordneten Knoten. Jeder dieser Zeiger enthält zusätzlich die Information, ob es sich um den linken (0) oder rechten (1) Kindknoten handelt. Das Verfahren arbeitet fortlaufend: Als Eingabe dient der aktuelle Huffman-Baum sowie das nächste zu kodierende Zeichen. Nach jedem Schritt wird das Gewicht des entsprechenden Knotens – also die Häufigkeit des Symbols – um eins erhöht. Dadurch kann sich die Reihenfolge der Knoten ändern: Sobald ein Knoten ein höheres Gewicht erreicht als das nächstgrößere Geschwisterpaar, werden die beiden Knoten vertauscht. Auf diese Weise bleibt der Baum stets in einer Form, die der optimalen Huffman-Struktur entspricht. Wird ein bisher unbekanntes Symbol eingelesen, so wird ein neuer Knoten für dieses Symbol eingefügt und in die bestehende Baumstruktur integriert. Die Kodierung erfolgt dann unmittelbar mit dem aktualisierten Baum, sodass die Methode immer konsistent bleibt. Das Dekodieren kann parallel erfolgen, da der Empfänger dieselbe Abfolge von Baumaktualisierungen nachvollzieht. Diese adaptive Methode wurde von Robert G. Gallager vorgeschlagen, der die theoretischen Grundlagen der Geschwister-Eigenschaft entwickelte und zeigte, dass sie eine eindeutige und optimale Baumstruktur garantiert, solange die Aktualisierungsregeln eingehalten werden\footnote{\cite{gallager2003variations} S.17–21: Adaptive Huffman Codes}. Donald E. Knuth verbesserte später Gallagers System, indem er effiziente Verfahren zur Verwaltung der Knoten und zum Umgang mit der Verringerung von Symbolzählern einführte, falls Zeichen aus dem Datenstrom entfernt werden\footnote{\cite{knuthDynamicHuffman} S.163–165: Introduction}. Das adaptive Huffman-System stellt somit eine effiziente Lösung für Echtzeit-Kompression dar, insbesondere in Szenarien, in denen die Symbolwahrscheinlichkeiten nicht im Voraus bekannt oder nicht konstant sind.

\subsection{Unterschied in der Effizienz zwischen statischen und adaptiven Kodierungssystemen}

Theoretisch gesehen sind statische Kodierungssysteme optimaler, aber nur wenn die Wahrscheinlichkeiten der Eingabesequenz genau bekannt sind und nicht über Zeit variieren. So sind in der Realität adaptive Verfahren meist praktischer und effizienter, da sie sich selbstständig gemäß der nachfolgenden Datenmenge anpassen. Bezüglich der Optimalität adaptiver Kodierungssysteme wurde durch mehrere Beispiele gezeigt, dass sie nur wenig von ihren statischen Versionen abweichen\footnote{\cite{cleary1984comparison} S.314: VI Summary}. So lohnt es sich in der Realität fast immer einen adaptiven statt einen statischen Kodierungssystem zu implementieren.